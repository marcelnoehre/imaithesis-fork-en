\documentclass[12pt,        % standard font size
  english,ngerman,          % german primary, english secondary
  paper=a4,                 % standard paper size
  captions=tablesignature,  % captions below tables
  listof=numbered,          % including lists of... in the ToC
  bibliography=totoc,       % references in ToC
  headings=small,           % size of headings
  headinclude=false,        % don't include page head in page layout
  footinclude=false,        % don't include page foot in page layout
  parskip=half-,            % space between paragraphs, no indentation
%   dotlessnumbers,         % no dots after chapter etc. numbers
% FIXME Adjust following 2 options for two-sided print, nr of pages
  oneside,                  % one-sided print
%   twoside,                % two-sided print
% FIXME Adjust the Binding CORrection to the binding and number of pages
% 5mm should be fine for project and seminar reports with between
% 30 and 60 pages when using plastic folders (Schnellhefter)
% For more information see the documentation for KOMA-classes
%   BCOR=15mm,                 % BCOR for thesis (hardcover)
%   BCOR=5mm,                 % BCOR for project reports (plastic folder)
%                             % No BCOR for seminar reports (plastic folder)
%   draft,                  % speed things up before final version
%   DIV=calc                % calculate the page layout
  DIV=12                    %  
%   DIV=classic                    %  
  ]{scrbook}                % KOMA script book class
% TODO font size 11 or 10, div?

%%%%%%%%%%%%%%%%%%%%%%%%%%%%%%%%%%%%%%%%%%%%%%%%%%%%%%%%%%%%
% Copyright 2013-2023 by Jörg Cassens <cassens@cs.uni-hildesheim.de>
%       and      2013 by Amelie Roenspieß (part of the example texts only)
%%%%%%%%%%%%%%%%%%%%%%%%%%%%%%%%%%%%%%%%%%%%%%%%%%%%%%%%%%%%
% LICENSE INFORMATION BEGINS
% 
% This is a sample LaTeX document intended for deliverables in media 
% informatics at the Institute for Computer Science, University of 
% Hildesheim, Germany. The purpose of this file is to serve as a 
% customizable template to suit your specific requirements.
% 
% As a user, you are granted the following permissions:
% 
% a) If you choose to use this file as a template (after removing
% the provided example texts), you are permitted to freely copy
% and modify it according to your needs, including the option to
% delete this copyright notice.
% 
% b) If you wish to modify this file and distribute it as a template, 
% you are granted permission to freely copy and modify it according 
% to your needs, including changing the example texts provided.
% 
% c) You may also distribute the unchanged file.
% 
% Please be aware that this collection of files contains logos
% of the University of Hildesheim. It is your responsibility
% to ensure that you have the necessary rights to use these logos. 
% If you intend to distribute the files outside the University of
% Hildesheim, you must remove the logos.
% 
% LICENSE INFORMATION ENDS
%%%%%%%%%%%%%%%%%%%%%%%%%%%%%%%%%%%%%%%%%%%%%%%%%%%%%%%%%%%%
% Adapt it where you see fit - you don't have to follow all
% idiosyncrasies of the default style, but are free to use
% typographically sane variations.
%
% You should be able to use it as it is, parts where you
% have to change things or decide on options are marked
% with comments like this:
% FIXME Example: change something here
%
% This template should only need packages available in a
% standard TeXlive install. Input encoding is UTF 8, please
% make sure your editor gets that right. We use pdfLaTeX
% and produce PDF directly, we do not go via DVI.
%
% Happy TeXing:)
%
% Please direct comments, feedback, patches to:
% 
%            cassens@cs.uni-hildesheim.de (jc)
% 
% gitlab-project at:
% 
%   https://www.uni-hildesheim.de/gitlab/cassensj/mithesis
% 
%
% ChangeLog
% 2010-09-23  project started for imis at uni of lübeck (jc)
% 2010-09-29  initial release (jc)
% 2010-09-30  bugfix imis.bst (jc)
% 2010-10-08  hacks for listings added (jc)
% 2010-11-16  changed titlepage as in new word template (jc)
% 2011-02-02  experimental workaround listings in german (jc)
% 2011-06-17  changed way of referencing weblinks (jc)
% 2011-07-29  removed broken package ltablex and use plain
%             tabularx. Beware: tabularx does not support
%             tables over several pages (thanks Niels for bug
%             report & Bjoern for debugging) (jc)
% 2011-12-22  changed Betreuer to Wissenschaftliche Begleitung
%             added some color to the listings (jc)
% 2012-03-08  major rework of content (mostly) and style to
%             reflect newest, reworked version of Word template
%             with a lot of new text to help you get going (jc)
% 2012-03-09  reworked listings and captions massively
%             now we rely on ccaptions.sty for a new 
%             imislistings float environment (jc)
% 2012-03-15  graphs and tables from data (jc)
% 2012-04-02  new title page, small changes to glossaries (jc)
% 2012-04-18  fixed small bug in Makefile resulting in a
%             wrong zipfile for publication - devel only (jc)
% 2012-11-09  hex color definitions to uppercase because some
%             installs seem to bark else - thanks Marcin (jc)
% 2012-12-07  Incorporated changes to support project reports
%             as well as theses, use a boolean to switch (jc)
% 2013-06-26  Initial import of project for use in Hildesheim (jc)
% 2013-07-31  Many simplifications, since we do not have to mimic
%             a given style we rely much more on the features of
%             TeX and use less packages
%             Content and form is adapted to what I think would
%             be a good starting point for theses in Hildesheim (jc)
% 2015-10-21  Small adjustments to formatting, preparing to 
%             use uni-gitlab for future development, addded scrhack
%             to make koma work better with listings.sty (jc)
% 2015-12-05  Refactoring to support a seminar template (jc)
% 2016-04-18  Changed font size to 12 to make it comparable to
%             new Microsoft Word template. The same number of
%             pages should have about the same number of characters.
%             However, this change makes the LaTeX-version look
%             less crisp. (jc)
% 2016-04-28  Clarified Copyright status and conditions of use (jc)
% 2017-03-28  Added example for citations with page numbers (jc)
% 2017-03-29  Moved hypersetup from setup.tex to seminar.tex (jc) 
% 2020-11-07  move from IMAI to IfI (jc)
% 2021-05-01  changes to logos (jc)
% 2023-05-16  added methods chapter (jc)
% 2023-05-25  license and use of generative AI clarified (jc)
% 2025-04-17  amended information on generative AI, removed the 
%             secondary web sources bibliography, fixed some warnings
%             caused by using deprecated commands (jc)
% 
%%%%%%%%%%%%%%%%%%%%%%%%%%%%%%%%%%%%%%%%%%%%%%%%%%%%%%%%%%%%

% most setup is done in a file of its own since it is shared
% by thesis and seminar
%%%%%%%%%%%%%%%%%%%%%%%%%%%%%%%%%%%%%%%%%%%%%%%%%%%%%%%%%%%%
% Copyright 2013-2025 by Jörg Cassens <cassens@cs.uni-hildesheim.de>
%
%%%%%%%%%%%%%%%%%%%%%%%%%%%%%%%%%%%%%%%%%%%%%%%%%%%%%%%%%%%%
% LICENSE INFORMATION BEGINS
% 
% This file is part of a collection of sample LaTeX documents
% intended as templates for deliverables in media  informatics
% at the  Institute for Computer Science, University of
% Hildesheim, Germany. 
% 
% As a user, you are granted the following permissions:
% 
% a) If you choose to use one of the templates provided, you are
% permitted to freely copy and modify this file according to your
% needs.
% 
% b) If you wish to modify this file and distribute it with one
% or more templates, you are granted permission to freely copy 
% and modify it according to your needs.
% 
% c) You may also distribute the unchanged file.
% 
% d) This file may also be distributed and/or modified
%
%    1. under the LaTeX Project Public License and/or
%    2. under the GNU Public License.
% 
% Please be aware that this collection of files contains logos
% of the University of Hildesheim. It is your responsibility
% to ensure that you have the necessary rights to use these logos. 
% If you intend to distribute the files outside the University of
% Hildesheim, you must remove the logos.
% 
% LICENSE INFORMATION ENDS
%%%%%%%%%%%%%%%%%%%%%%%%%%%%%%%%%%%%%%%%%%%%%%%%%%%%%%%%%%%%
\makeatletter
\renewcommand*\bib@heading{%
  \section*{\bibname}%
  \@mkboth{\bibname}{\bibname}%
}
\makeatother

%%%%%%%%%%%%%%%%%%%%%%%%%%%%%%%%%%%%%%%%%%%%%%%%%%%%%%%%%%%%
% General packages
%%%%%%%%%%%%%%%%%%%%%%%%%%%%%%%%%%%%%%%%%%%%%%%%%%%%%%%%%%%%
\usepackage{scrhack}
\usepackage{iftex}
\usepackage{babel}

%%%%%%%%%%%%%%%%%%%%%%%%%%%%%%%%%%%%%%%%%%%%%%%%%%%%%%%%%%%%
% Fonts
%%%%%%%%%%%%%%%%%%%%%%%%%%%%%%%%%%%%%%%%%%%%%%%%%%%%%%%%%%%%
  \ifXeTeX
    % XeLaTeX (system fonts via fontspec)
    \usepackage{fontspec}
    \usepackage{unicode-math}
    \setmainfont {TeX Gyre Pagella}[Scale = 1.05]
    \setromanfont{TeX Gyre Pagella}[Scale = 1.05]
    \setmathfont {TeX Gyre Pagella Math}
    \setsansfont {Source Sans Pro}
    \setmonofont {Source Code Pro}
    \defaultfontfeatures{Ligatures=TeX}
  \else\ifLuaTeX
    % LuaLaTeX (system fonts via fontspec, same syntax as XeTeX)
    \usepackage{fontspec}
    \usepackage{unicode-math}
    \setmainfont {TeX Gyre Pagella}[Scale = 1.05]
    \setromanfont{TeX Gyre Pagella}[Scale = 1.05]
    \setmathfont {TeX Gyre Pagella Math}
    \setsansfont {Source Sans Pro}
    \setmonofont {Source Code Pro}
    \defaultfontfeatures{Ligatures=TeX}
  \else
    % pdfLaTeX (fallback to traditional fonts)
    \usepackage[utf8]{inputenc}
    \usepackage[T1]{fontenc}
    \usepackage[scaled=1.05]{newpxtext}
    \usepackage{newpxmath}
    \usepackage{sourcecodepro}
    \usepackage{sourcesanspro}
  \fi\fi

%%%%%%%%%%%%%%%%%%%%%%%%%%%%%%%%%%%%%%%%%%%%%%%%%%%%%%%%%%%%
% Look & Feel
%%%%%%%%%%%%%%%%%%%%%%%%%%%%%%%%%%%%%%%%%%%%%%%%%%%%%%%%%%%%
\usepackage{scrlayer-scrpage}
\usepackage{url}
\usepackage{tabularx}
\usepackage{enumitem}
\usepackage{pifont}
\usepackage{listings}

%%%%%%%%%%%%%%%%%%%%%%%%%%%%%%%%%%%%%%%%%%%%%%%%%%%%%%%%%%%%
% Graphics
%%%%%%%%%%%%%%%%%%%%%%%%%%%%%%%%%%%%%%%%%%%%%%%%%%%%%%%%%%%%
\usepackage{graphicx}
\usepackage[dvipsnames,usenames]{xcolor}
\usepackage{tikz}
\usepackage{pgfplots}
\usepackage{pgfplotstable}
\pgfplotsset{compat=1.15}

%%%%%%%%%%%%%%%%%%%%%%%%%%%%%%%%%%%%%%%%%%%%%%%%%%%%%%%%%%%%
% Bib
%%%%%%%%%%%%%%%%%%%%%%%%%%%%%%%%%%%%%%%%%%%%%%%%%%%%%%%%%%%%
\usepackage{natbib}

%%%%%%%%%%%%%%%%%%%%%%%%%%%%%%%%%%%%%%%%%%%%%%%%%%%%%%%%%%%%
% Colors
%%%%%%%%%%%%%%%%%%%%%%%%%%%%%%%%%%%%%%%%%%%%%%%%%%%%%%%%%%%%
\definecolor{lightgray}{gray}{.5}
%%%%%%%%%%%%%%%%%%%%%%%%%%%%%%%%%%%%%%%%%%%%%%%%%%%%%%%%%%%%
\definecolor{tango_orange_dark}      {HTML}{CE5C00}
\definecolor{tango_orange_medium}    {HTML}{F57900}
\definecolor{tango_orange_light}     {HTML}{FCAF3E}
\definecolor{tango_chocolate_dark}   {HTML}{8F2902}
\definecolor{tango_chocolate_medium} {HTML}{C17D11}
\definecolor{tango_chocolate_light}  {HTML}{E9B96E}
\definecolor{tango_chameleon_dark}   {HTML}{4E9A06}
\definecolor{tango_chameleon_medium} {HTML}{73D216}
\definecolor{tango_chameleon_light}  {HTML}{8AE234}
\definecolor{tango_skyblue_dark}     {HTML}{204A87}
\definecolor{tango_skyblue_medium}   {HTML}{3465A4}
\definecolor{tango_skyblue_light}    {HTML}{729FCF}
\definecolor{tango_plum_dark}        {HTML}{5C3566}
\definecolor{tango_plum_medium}      {HTML}{75507B}
\definecolor{tango_plum_light}       {HTML}{AD7FA8}
\definecolor{tango_scarletred_dark}  {HTML}{A40000}
\definecolor{tango_scarletred_medium}{HTML}{CC0000}
\definecolor{tango_scarletred_light} {HTML}{EF2929}
\definecolor{tango_aluminimum_1}     {HTML}{EEEEEC}
\definecolor{tango_aluminimum_2}     {HTML}{D3D7CF}
\definecolor{tango_aluminimum_3}     {HTML}{BABDB6}
\definecolor{tango_aluminimum_4}     {HTML}{888A85}
\definecolor{tango_aluminimum_5}     {HTML}{555753}
\definecolor{tango_aluminimum_6}     {HTML}{2E3436}
%%%%%%%%%%%%%%%%%%%%%%%%%%%%%%%%%%%%%%%%%%%%%%%%%%%%%%%%%%%%
\definecolor{unihi_red}   {HTML}{D7001C} % Uni 
\definecolor{unihi_ruby}  {HTML}{B4152B} % FB I
\definecolor{unihi_orange}{HTML}{EC870E} % FB II
\definecolor{unihi_blue}  {HTML}{0186BC} % FB III
\definecolor{unihi_green} {HTML}{467A40} % FB IV

%%%%%%%%%%%%%%%%%%%%%%%%%%%%%%%%%%%%%%%%%%%%%%%%%%%%%%%%%%%%
% Other
%%%%%%%%%%%%%%%%%%%%%%%%%%%%%%%%%%%%%%%%%%%%%%%%%%%%%%%%%%%%
\usepackage[
  colorlinks=true,
  linkcolor=tango_chocolate_dark,
  menucolor=tango_chocolate_dark,
  citecolor=tango_skyblue_dark,
  urlcolor=tango_plum_dark,
]{hyperref}
\recalctypearea

%%%%%%%%%%%%%%%%%%%%%%%%%%%%%%%%%%%%%%%%%%%%%%%%%%%%%%%%%%%%
% Set up header and footer
%%%%%%%%%%%%%%%%%%%%%%%%%%%%%%%%%%%%%%%%%%%%%%%%%%%%%%%%%%%%
\pagestyle{scrheadings}
\clearpairofpagestyles
\renewcommand{\headfont}{\normalfont\rmfamily\itshape}
\ohead{\headmark}
\ofoot[\pagemark]{\pagemark}


%%%%%%%%%%%%%%%%%%%%%%%%%%%%%%%%%%%%%%%%%%%%%%%%%%%%%%%%%%%%
% General changes to formatting
%%%%%%%%%%%%%%%%%%%%%%%%%%%%%%%%%%%%%%%%%%%%%%%%%%%%%%%%%%%%
\lstset{
  keywordstyle=\color{red!40!black},
  commentstyle=\itshape\color{green!40!black},
  stringstyle=\color{blue!50!black},
  basicstyle=\ttfamily,
  frame=single,
  language=sparql,
  captionpos=b,
  nolol=false
}

\addtokomafont{caption}{\footnotesize}
\setkomafont{sectioning}{\rmfamily\bfseries}
\urlstyle{same}

%%%%%%%%%%%%%%%%%%%%%%%%%%%%%%%%%%%%%%%%%%%%%%%%%%%%%%%%%%%%
% float control
%%%%%%%%%%%%%%%%%%%%%%%%%%%%%%%%%%%%%%%%%%%%%%%%%%%%%%%%%%%%
\setcounter{topnumber}{2}
\setcounter{bottomnumber}{9}
\setcounter{totalnumber}{20}
\setcounter{dbltopnumber}{9}
\renewcommand{\topfraction}{0.85}
\renewcommand{\bottomfraction}{0.7}
\renewcommand{\textfraction}{0.15}
\renewcommand{\floatpagefraction}{0.7}
\renewcommand{\dbltopfraction}{.7}
\renewcommand{\dblfloatpagefraction}{.7}

%%%%%%%%%%%%%%%%%%%%%%%%%%%%%%%%%%%%%%%%%%%%%%%%%%%%%%%%%%%%
% some commands necessary for the template
%%%%%%%%%%%%%%%%%%%%%%%%%%%%%%%%%%%%%%%%%%%%%%%%%%%%%%%%%%%%
\newcommand{\imaicomment}[1]{{\itshape\color{tango_aluminimum_4} #1}}
\newcommand{\migenaiverb}[1]{{\color{tango_orange_dark} #1}}
\newcommand{\migenaimod}[1]{{\color{tango_chameleon_dark} #1}}
\newcommand{\migenaiused}{{\color{tango_chameleon_dark}\raisebox{.125ex}{\scalebox{0.8}{\ding{70}}}\hspace{-0.2em}\raisebox{.8ex}{\scalebox{0.5}{\ding{70}}}\hspace{-0.4em}\raisebox{-0.1ex}{\scalebox{0.5}{\ding{70}}}}}
\newcommand*{\imaitexttt}[1]{\texttt{%
    \fontdimen2\font=0.4em% interword space
    \fontdimen3\font=0.2em% interword stretch
    \fontdimen4\font=0.1em% interword shrink
    \fontdimen7\font=0.1em% extra space
    \hyphenchar\font=`\-% allowing hyphenation
    #1}}


%%%%%%%%%%%%%%%%%%%%%%%%%%%%%%%%%%%%%%%%%%%%%%%%%%%%%%%%%%%%
% let the fun begin
%%%%%%%%%%%%%%%%%%%%%%%%%%%%%%%%%%%%%%%%%%%%%%%%%%%%%%%%%%%%
\begin{document}

% stuff before real content, no chapter marks, roman page numbers
\frontmatter

% % the cite commands will generate entries in the index, if used
% \citeindextrue

%%%%%%%%%%%%%%%%%%%%%%%%%%%%%%%%%%%%%%%%%%%%%%%%%%%%%%%%%%%%
% We don't use a custom \maketitle for now
% so we have to make our own title page
%%%%%%%%%%%%%%%%%%%%%%%%%%%%%%%%%%%%%%%%%%%%%%%%%%%%%%%%%%%%

\begin{titlepage}
% FIXME Title, author and subject for PDF metadata
  \hypersetup{%
    pdftitle={media informatics seminar template},
    pdfauthor={Jörg Cassens},
    pdfsubject={Template, examples and explanations for media informatics seminar reports.}
  }%
  \begin{center}
    \includegraphics[width=2.5cm]{figures/unihi-logo.pdf}\\
%     \includegraphics[width=5cm]{figures/unihi-ifi-logo-en.pdf}\\
%     \includegraphics[width=5cm]{figures/unihi-ifi-logo-de.pdf}\\

    \vfill

    {
      \Large
      \bfseries
%     FIXME your real title in German
      Titel der Arbeit\\    
      Titel Zeile 2
      
      \vspace{0.25cm}
      \normalsize
      \mdseries
%     FIXME your real title in English
      \selectlanguage{english} % switching to english
        Title of the Thesis in English\\
        Second Line English Title
      \selectlanguage{ngerman}
    }

    \vfill

    {
      \normalsize

%     FIXME your type of thesis
      Seminararbeit

%     FIXME Chose your study program:
      im Rahmen des Studiengangs\\
      Informationsmanagement und Informationstechnologie\\
%     Wirtschaftsinformatik\\    
      der Universität Hildesheim

      \vfill

%     FIXME your real name - do not add the student number
      Vorgelegt von\\
      Vor- und Zuname des/der Studierenden

      \vfill

      Prüfer\\
      Dr. Jörg Cassens
  
      \vfill
      
%     FIXME real date of delivery
      Hildesheim, \today
    }
      
      \vfill
      
    {
      \footnotesize 

      \vfill

      Fachbereich IV -- Mathematik, Naturwissenschaften, Wirtschaft und Informatik\\      
      Institut für Informatik
    }
  \end{center}
\end{titlepage}
%%%%%%%%%%%%%%%%%%%%%%%%%%%%%%%%%%%%%%%%%%%%%%%%%%%%%%%%%%%%

%%%%%%%%%%%%%%%%%%%%%%%%%%%%%%%%%%%%%%%%%%%%%%%%%%%%%%%%%%%%
\chapter*{Kurzfassung}
% \thispagestyle{empty}
%%%%%%%%%%%%%%%%%%%%%%%%%%%%%%%%%%%%%%%%%%%%%%%%%%%%%%%%%%%%
\imaicomment{
  Eine kurze Beschreibung der Arbeit

  Generelle Hinweise:

  \begin{itemize}
    \item Die grauen, kursiven Kommentare sind Hinweise zum Inhalt,
      der schwarze Text ist beispielhafter Inhalt.
    \item Dieses Dokument ist für einseitigen Druck formatiert; wenn
      zweiseitig gedruckt werden soll, muß das Seitenformat (Kopf, 
      Seitenzahlen) entsprechend angepaßt werden.
    \item Auf Abbildungen/Tabellen möglichst im Text vor der
      Abbildung verweisen.
    \item Abbildungen sollten nach Möglichkeit so groß dargestellt
      sein, dass auch die Texte gut lesbar sind (in der Regel mindestens
      in der Schriftgröße von Fußnoten); es sei denn die Texte sind
      völlig bedeutungslos und nur die Struktur oder das Gesamtbild
      sind von Bedeutung.
    \item Falls farbige Abbildungen verwendet werden sollte sichergestellt werden, daß
      diese auch in Schwarz-Weiß gut erkennbar sind.
    \item Tabellen sollten zweckmäßig und übersichtlich sein: Vermeidung
      unnötiger Linien, Einsatz von Farben nur, wenn sie eine Bedeutung hat oder
      der Übersichtlichkeit dient.
    \item Falls generative KI-Systeme benutzt werden sind die Anmerkungen
      zur Formatierung in Kapitel \ref{chapter:introduction} zu beachten.
  \end{itemize}
  
  Ein Teil der Beispieltexte und Erläuterungen wurde von Amelie
  Roenspieß erstellt.
  }

Mit diesem Dokument wird eine Gestaltungsempfehlung für das Erstellen von
Seminararbeiten in der Medieninformatik am Institut für Mathematik und 
Angewandte Informatik der Universität Hildesheim vorgelegt.

Diese Vorlage unterscheidet sich von der Vorlage für Bachelor- und
Masterarbeiten neben Unterschieden in der inhaltlichen Gliederung vor allem
dadurch, daß kein Glossar und kein Abkürzungsverzeichnis angelegt werden.
Falls diese Teil der Seminararbeit sein sollen sollte die Vorlage für
Abschlußarbeiten verwendet und der Inhalt entsprechend angepaßt werden.

\vfill

%%%%%%%%%%%%%%%%%%%%%%%%%%%%%%%%%%%%%%%%%%%%%%%%%%%%%%%%%%%%
\section*{Schlüsselwörter} Medieninformatik, Interaktive Medien
%%%%%%%%%%%%%%%%%%%%%%%%%%%%%%%%%%%%%%%%%%%%%%%%%%%%%%%%%%%%


\selectlanguage{english} % switching to english
%%%%%%%%%%%%%%%%%%%%%%%%%%%%%%%%%%%%%%%%%%%%%%%%%%%%%%%%%%%%
\chapter*{Abstract}
% \thispagestyle{empty}
%%%%%%%%%%%%%%%%%%%%%%%%%%%%%%%%%%%%%%%%%%%%%%%%%%%%%%%%%%%%

\imaicomment{
  A short description of the thesis in English.

  General instructions:

  \begin{itemize}
    \item The grey, italic comments are instructions; the black
      text is exemplary content.
    \item This document is formatted for single sided printing;
      if you want to print double sided please remember to adjust
      the page numbers and headers correspondingly.
    \item Try to refer to figures and tables in the preceding text.
    \item Make sure coloured figures are still informative in black
      and white.
    \item Make sure text within figures is readable (usually at least
      in the size of footnotes) unless it is irrelevant and the figure
        is only supposed to show a structure or give a general impression.
    \item Tables ought to be functional: Avoid unnecessary lines and
      colour, unless they convey additional information or help structure
      the table.
    \item Obey the citation guidelines for scientific work.
    \item If generativ AI-systems are being used you will have to follow
      the formatting guideline in Chapter \ref{chapter:introduction}.
  \end{itemize}
  
  The exemplary text and guidelines are partially written by Amelie Roenspieß.
  }
  
  This document serves as a design guideline for writing seminar reports in
  Media Informatics at the Institute for Mathematics and Applied Informatics,
  University of Hildesheim.

  Compared to the template for bachelor and master theses, the main difference
  of this template, besides the differences in content, is that we will
  generate neither a glossary nor a list of abbreviations. If you intend to use
  these in your seminar delivery, just use the template designed for theses.

\vfill

%%%%%%%%%%%%%%%%%%%%%%%%%%%%%%%%%%%%%%%%%%%%%%%%%%%%%%%%%%%%
\section*{Keywords} Media Informatics, Interactive Media
%%%%%%%%%%%%%%%%%%%%%%%%%%%%%%%%%%%%%%%%%%%%%%%%%%%%%%%%%%%%
\selectlanguage{ngerman} % and back to german

% if using tocloft.sty, we need an explicit clearpage
\clearpage
% we want chapters and sections in the toc, then generate it
\setcounter{tocdepth}{1}
\tableofcontents

% now we have real content, with section marks and egyptian numerals
\mainmatter

%%%%%%%%%%%%%%%%%%%%%%%%%%%%%%%%%%%%%%%%%%%%%%%%%%%%%%%%%%%%
%%%%%%%%%%%%%%%%%%%%%%%%%%%%%%%%%%%%%%%%%%%%%%%%%%%%%%%%%%%%
\chapter{Einleitung}\label{chapter:introduction}
%%%%%%%%%%%%%%%%%%%%%%%%%%%%%%%%%%%%%%%%%%%%%%%%%%%%%%%%%%%%

\imaicomment{
  Einführung und Motivation des Themas

  Kurze Einleitung zu den Unterkapiteln
  
  Darstellung der Problemstellung, Hinführung zum Thema
  }

Das Erstellen einer Seminararbeit ist ein elementarer Bestandteil forschungsorientierte Seminare. Da es hierfür sehr viele verschiedene  Gestaltungsmöglichkeiten gibt, kann es für Studierende schwierig sein, eine für ihre Aufgabe geeignete Vorlage zu finden. Es bietet sich daher  an, ein System zu entwickeln, welches Studierende bei der Auswahl eines  gestalterischen Rahmens für ihre Arbeit unterstützt.

In diesem Kapitel wird ein einleitender Überblick über die Ziele der vorliegenden Arbeit, ferner werden einige grundlegende Begriffe eingeführt. Im folgenden Kapitel \ref{chapter:main} werden Theorien, Methoden und Vorgehensweisen zum Bereich der Vorlagenerstellung vorgestellt. Im \ref{chapter:example}. Kapitel wird eine beispielhafte Realisierung zusammen mit der eingesetzten Technik vorgestellt. Im letzten Kapitel werden eine Zusammenfassung der Arbeit sowie eine kritische Einschätzung gegeben.

%%%%%%%%%%%%%%%%%%%%%%%%%%%%%%%%%%%%%%%%%%%%%%%%%%%%%%%%%%%%
\section{Generative KI}
%%%%%%%%%%%%%%%%%%%%%%%%%%%%%%%%%%%%%%%%%%%%%%%%%%%%%%%%%%%%

\imaicomment{
  Richtlinien für den Einsatz generativer KI
  
  Dieses Unterkapitel wird in der Regel nicht Teil der Arbeit sein
  }
  
Im Rahmen der Lehre in der Medieninformatik ist die Nutzung generativer KI-Systeme \textit{ausdrücklich erlaubt}, sofern sie verantwortungsvoll eingesetzt und transparent dokumentiert wird. Diese Richtlinien sollen Ihnen helfen, generative KI als Werkzeug im wissenschaftlichen Arbeitsprozess sinnvoll einzusetzen und gleichzeitig akademische Integrität zu wahren. Die Grundprinzipien dabei sind:
    
\begin{itemize}
  \item Sie tragen die volle \textit{wissenschaftliche Verantwortung} für alle in Ihren Ausarbeitungen und Präsentationen enthaltenen Inhalte, auch wenn sie mit Unterstützung von KI-Systemen erstellt wurden.
  \item Generative KI kann Fakten erfinden (halluzinieren), Verzerrungen (Bias) verstärken und sachlich falsch sein. \textit{Überprüfen Sie daher alle durch KI generierten Inhalte kritisch}.    
  \item KI-Tools sind \textit{keine zitierfähigen Quellen}. Wenn Sie mit KI recherchieren, müssen Sie stets die ursprünglichen wissenschaftlichen Quellen identifizieren, selber aufbereiten und korrekt zitieren.
  \item Die KI-Nutzung sollte als \textit{Unterstützung des eigenen Denkprozesses} verstanden werden, nicht als Ersatz.
\end{itemize}

Genutzte generative KI-Systeme \textbf{müssen} am Ende einer jeden schriftlichen Arbeit aufgelistet werden. Mindestens die verwendeten Systeme und ihre Version sind darzulegen. Diese Auflistung kann im besonderen durch die eine tabellarische Dokumentation geschehen, wie im Beispiel auf Seite \pageref{sec:ai_use} gezeigt.

 
Darüber hinaus \textbf{können} Sie eine farbliche Kennzeichnung im Text benutzen:
  
\migenaimod{%
    Diese Farbe kennzeichnet Textabschnitte, bei denen Ausgaben einer generativen KI (wie zum Beispiel GPT) verwendet und erheblich verändert und zu eigenen gemacht wurden. Dies umfasst kritische Textanalyse, Hinzufügen eigener Gedanken, Paraphrasierung und substanzielle Überarbeitung. Diese Kennzeichnung gilt auch, wenn eine generative KI einen selbst formulierten Text maßgeblich überarbeitet hat.%
  }

\migenaiverb{%
    Diese Farbe kennzeichnet Textabschnitte, in denen Ausgaben einer generativen KI übernommen und nicht substantiell überarbeitet wurden.%
  }
  
\textit{Nicht gekennzeichnet} wird die Verwendung von KI-Tools für das Copy-Editing. Unter Copy-Editing verstehen wir KI-unterstützte Verbesserungen an menschlich erstellten Texten für Lesbarkeit und Stil sowie zur Korrektur von Fehlern in Grammatik, Rechtschreibung, Zeichensetzung und Tonfall. Diese KI-unterstützten Verbesserungen können Formulierungs- und Formatierungsänderungen umfassen, beinhalten jedoch keine eigenständige redaktionelle Arbeit oder autonome Inhaltserstellung.

Ebenso muss die Verwendung von Übersetzungsdiensten nicht offengelegt werden.
  
\textit{Wichtig:} Bei allen KI-unterstützten Arbeiten bleibt die menschliche Verantwortung für den finalen Text bestehen. Die Studierenden müssen sicherstellen, dass die Inhalte korrekt sind und ihre ursprüngliche Arbeit angemessen widerspiegeln.
    
\textit{Die KI-Nutzung selbst wird nicht bewertet}, sondern nur die Qualität der wissenschaftlichen Arbeit und die transparente Dokumentation der Nutzung. Die fehlende Offenlegung der KI-Nutzung kann jedoch als Täuschungsversuch gewertet werden.

%%%%%%%%%%%%%%%%%%%%%%%%%%%%%%%%%%%%%%%%%%%%%%%%%%%%%%%%%%%%
\section{Ziele der Arbeit}\label{sec:goals}
%%%%%%%%%%%%%%%%%%%%%%%%%%%%%%%%%%%%%%%%%%%%%%%%%%%%%%%%%%%%

\imaicomment{Beschreibung und Begründung der Ziele und die Relevanz des Themas}

In dieser Arbeit werden interaktive Systeme zur Bereitstellung von Dokumentvorlagen für Abschlußarbeiten untersucht. Die aktuelle Forschung wird vorgestellt und eingeordnet.

%%%%%%%%%%%%%%%%%%%%%%%%%%%%%%%%%%%%%%%%%%%%%%%%%%%%%%%%%%%%
\section{Definitionen}\label{sec:definitions}
%%%%%%%%%%%%%%%%%%%%%%%%%%%%%%%%%%%%%%%%%%%%%%%%%%%%%%%%%%%%

\imaicomment{Grundlegende Begriffe werden eingeführt}

Unter Empfehlungssystemen (englisch: recommender systems) versteht man Systeme, die Benutzerinnen und Benutzern Vorschläge aufgrund bekannter Vorlieben machen \cite{herczeg-2009-software_ergonomie}. Bei Online-Handelsplattformen wie amazon wird dazu beispielsweise ausgewertet, welche Produkte vorher gekauft wurden, um neue Produkte zu empfehlen.

%%%%%%%%%%%%%%%%%%%%%%%%%%%%%%%%%%%%%%%%%%%%%%%%%%%%%%%%%%%%
%%%%%%%%%%%%%%%%%%%%%%%%%%%%%%%%%%%%%%%%%%%%%%%%%%%%%%%%%%%%
\chapter{Methodik}\label{sec:approach}
%%%%%%%%%%%%%%%%%%%%%%%%%%%%%%%%%%%%%%%%%%%%%%%%%%%%%%%%%%%%

\imaicomment{
  Überblick zur Vorgehensweise bei der Bearbeitung des Themas
  }
  
Es wurde eine Literaturrecherche in Form einer Scroping Review (Literaturübersicht) durchgeführt. Eine Scoping Review dient dem Zweck, die vorhandene Literatur zu einer bestimmten Forschungsfrage zu identifizieren. Er kann auch Konzepte in der Literatur klären und Wissenslücken definieren. Im Gegensatz zu systematischen Übersichtsartikeln haben Literaturübersichten nicht das Ziel, ein kritisch bewertetes und synthetisiertes Ergebnis auf eine bestimmte Frage zu liefern, sondern sie sollen vielmehr einen Überblick oder eine Landkarte des vorhandenen Evidenzmaterials liefern \citep{Munn_ea-2018-scoping_reivews}.

%%%%%%%%%%%%%%%%%%%%%%%%%%%%%%%%%%%%%%%%%%%%%%%%%%%%%%%%%%%%
%%%%%%%%%%%%%%%%%%%%%%%%%%%%%%%%%%%%%%%%%%%%%%%%%%%%%%%%%%%%
\chapter{Theorie}\label{chapter:main}
%%%%%%%%%%%%%%%%%%%%%%%%%%%%%%%%%%%%%%%%%%%%%%%%%%%%%%%%%%%%

\imaicomment{
  Die Kernaussagen des Basistextes werden vorgestellt
  
  Theorien, Methoden und Vorgehensweisen werden erläutert
  
  In der Regel ist es notwendig, weitere Literatur zu Rate zu ziehen, um das gesagte einordnen zu können
  
  Der Name dieses Kapitels, ``Theorie'', ist hier nur Platzhalter und sollte durch eine aussagekräftige Beschreibung des Inhaltes ersetzt werden.
  }

%%%%%%%%%%%%%%%%%%%%%%%%%%%%%%%%%%%%%%%%%%%%%%%%%%%%%%%%%%%%
\section{Stand von Wissenschaft und Technik}\label{sec:state_of_art}
%%%%%%%%%%%%%%%%%%%%%%%%%%%%%%%%%%%%%%%%%%%%%%%%%%%%%%%%%%%%

\imaicomment{
  Literatur-Recherche und Erwähnung anderer wichtiger Arbeiten zum Thema
  
  Darstellung des ``State of the Art'', kurze Vorstellung ähnlicher Arbeiten; bei ausführlichen Beschreibungen für den ``State of the Art'' kann statt des Unterkapitels auch ein eigenes Kapitel mit dem Titel ``Verwandte Arbeiten'' oder ``Stand der Technik'' sinnvoll sein

  Zitiert werden soll in der Arbeit wie in den Beispieltexten gezeigt.

  Bei Tabellen sollte auf unnötige Linien und Farbgebung verzichtet werden (übersichtlicher).
 }

Diverse \LaTeX-Dokumentvorlagen für das Verfassen von Abschlußarbeiten werden
beispielsweise im Katalog des CTAN
bereitgestellt\footnote{\href{https://www.ctan.org/}{www.ctan.org}}.

\begin{table}[ht]
  % We want some more white space in this table
  % First the column separation
  \renewcommand{\tabcolsep}{3mm}
  % Then the columns
  \renewcommand{\arraystretch}{1.5}
  \begin{center}
    \begin{tabular}{llll}
                              & \textbf{Vorteile}  & \textbf{Nachteile} & \textbf{Anmerkungen}\\
      \hline
      \textbf{Microsoft Word} & leicht erlernbar   & kostenpflichtig    & \\
      \textbf{\LaTeX}         & Schönes Satzbild   & lernintensiv       & erweiterbar \\
      \textbf{Libreoffice}    & leicht erlernbar   &                    & frei verfügbar \\
    \end{tabular}
    \caption{Vergleich von Textsystemen zur Erstellung von Abschlußarbeiten.}
    \label{table:typesetting}
  \end{center}
\end{table}%

Bei der Erstellung von Abschlußarbeiten ist es für Studierende nicht immer deutlich ersichtlich
an welchen Vorgaben sie sich orientieren müssen. Wie schon \citet[Seite 19]{bringhurst-2005-elements_typographic_style} erwähnt ist auch die Typographie ein schwieriger Aspekt \citep{willberg-2008-wegweiser_schrift}.

% abcdefghijklmnopqrstuvwxyzabcdefghijklmnopqrstuvwxyzabcdefghijklmnopqrstuvwxyz

%%%%%%%%%%%%%%%%%%%%%%%%%%%%%%%%%%%%%%%%%%%%%%%%%%%%%%%%%%%%
\section{Kernaussagen}\label{sec:main_points}
%%%%%%%%%%%%%%%%%%%%%%%%%%%%%%%%%%%%%%%%%%%%%%%%%%%%%%%%%%%%

\imaicomment{
  In diesem Abschnitt wird erläutert, was die vorgestellten Arbeiten vom Stand der Wissenschaft und Technik unterscheidet, inwiefern sie etwas neues darstellen
  
  Neben dem vorgegeben Basistext sollten auch weitere Wissenschaftliche Quellen einbezogen werden
  }

%%%%%%%%%%%%%%%%%%%%%%%%%%%%%%%%%%%%%%%%%%%%%%%%%%%%%%%%%%%%
%%%%%%%%%%%%%%%%%%%%%%%%%%%%%%%%%%%%%%%%%%%%%%%%%%%%%%%%%%%%
\chapter{Praxis}\label{chapter:example}
%%%%%%%%%%%%%%%%%%%%%%%%%%%%%%%%%%%%%%%%%%%%%%%%%%%%%%%%%%%%

\imaicomment{
  Beschreibung einer evtl. Realisierung (Hardware/Software) und einer evtl. Evaluation
  
  Struktur dieses Kapitel kann je nach den zugrundeliegenden Texten unterschiedlich gestaltet werden
  
  Der Name dieses Kapitels, ``Praxis'', ist hier nur Platzhalter und sollte durch eine aussagekräftige Beschreibung des Inhaltes ersetzt werden.
  }

%%%%%%%%%%%%%%%%%%%%%%%%%%%%%%%%%%%%%%%%%%%%%%%%%%%%%%%%%%%%
\section{Realisierung}\label{sec:implementation}
%%%%%%%%%%%%%%%%%%%%%%%%%%%%%%%%%%%%%%%%%%%%%%%%%%%%%%%%%%%%

\imaicomment{Beschreibung beispielhafter Systeme}

\begin{lstlisting}[%
  caption={SPARQL-Abfrage nach Personen.},
  label={lst:person }]
PREFIX foaf: <http://xmlns.com/foaf/0.1/>
SELECT ?name ?email
WHERE {
  ?person a foaf:Person.
  ?person foaf:name ?name.
  ?person foaf:mbox ?email.
}
\end{lstlisting}%

%%%%%%%%%%%%%%%%%%%%%%%%%%%%%%%%%%%%%%%%%%%%%%%%%%%%%%%%%%%%
\section{Evaluation}\label{sec:eval}
%%%%%%%%%%%%%%%%%%%%%%%%%%%%%%%%%%%%%%%%%%%%%%%%%%%%%%%%%%%%

\imaicomment{Beschreibung der in den referierten Texten beschriebenen Evaluation}

%%%%%%%%%%%%%%%%%%%%%%%%%%%%%%%%%%%%%%%%%%%%%%%%%%%%%%%%%%%%
\subsection{Vorgehen und Methoden}\label{sec:eva_approach}
%%%%%%%%%%%%%%%%%%%%%%%%%%%%%%%%%%%%%%%%%%%%%%%%%%%%%%%%%%%%

\imaicomment{
  Evaluationen können auf unterschiedliche Art und Weise erfolgen
    
  Beschreibung des von den Autorinnen und Autoren gewählten Vorgehens:

  \begin{itemize}
    \item Beschreibung der eingesetzten Methoden/Instrumente (Quellenangabe bei publizierten Fragebögen)
    \item Beschreibung der Untersuchungssituation/des Versuchsablaufs
    \item Beschreibung der Stichprobe und ihrer Gewinnung
  \end{itemize}
  }

%%%%%%%%%%%%%%%%%%%%%%%%%%%%%%%%%%%%%%%%%%%%%%%%%%%%%%%%%%%%
\subsection{Ergebnisse}\label{sec:eva_results}
%%%%%%%%%%%%%%%%%%%%%%%%%%%%%%%%%%%%%%%%%%%%%%%%%%%%%%%%%%%%

\imaicomment{
  Welche Ergebnisse brachte die Evaluierung und was ist davon zu halten\ldots

  Hier helfen Tabellen (Achsen erläutern) und Grafiken bei der Vermittlung von Sachverhalten

  Die verwendeten statistischen Verfahren sind zu benennen (z.B.\ t-Test für unabhängige 
  Stichproben) und die jeweiligen Kennwerte anzugeben
  }

% this defines some evaluation results in a table
% we use it to generate a table and a graph
% by this, we only have to change data in one place
\pgfplotstableread{
  F   M-A   SD-A  M-B   SD-B
  AA  5.27  0.69  3.83  0.48 
  SB  3.92  0.96  6.33  0.35 
  EK  5.88  1.10  6.10  0.56 
  LF  4.54  1.76  4.24  1.13 
  SK  5.75  0.92  6.90  0.67 
  FT  4.56  0.93  3.00  0.98 
  IK  4.61  0.72  5.86  0.82 
}\empiricaldata

% % % pdfLaTeX can use pdf, png, jpeg, etc... or tikz:
\begin{figure}[htb]
  \begin{center}  
% % % you can import the graph from a graphics file...
% %     \includegraphics[width=.8\textwidth]{figures/evaluation-graph}
% % ... but we generate the graph from the data entered  
    \begin{tikzpicture}
      \begin{axis}[ybar,
          ymax=8,
          enlargelimits=0.1,
          ylabel={Mittelwert},
          symbolic x coords={AA,SB,EK,LF,SK,FT,IK},
          width=.6\textwidth,
          ymajorgrids=true,
          legend pos=outer north east]
        \addplot+[draw=tango_skyblue_dark,
            fill=tango_skyblue_light,
            error bars/.cd,
            y dir=both,
            y explicit,
            error bar style={color=black}]
          table[x=F,
            y=M-A,
            y error=SD-A]
          {\empiricaldata};
        \addplot+[draw=tango_orange_dark,
            fill=tango_orange_light,
            error bars/.cd,
            y dir=both,
            y explicit,
            error bar style={color=black},]
          table[x=F,
            y=M-B,
            y error=SD-B]
          {\empiricaldata};
        \legend{~Gruppe A,~Gruppe B}
      \end{axis}
    \end{tikzpicture}
  \end{center}
  \caption{Mittelwerte und Standardabweichungen der Faktoren. Ermittelt mit Hilfe eines Fragebogens ISONORM 9241/110-S \protect\citep{prumper-2012-isonorm-fragebogen} für Gruppen A und B (N=20).}
  \label{fig:eval}
\end{figure}

Bei Gruppe A (M=5.27) ist die ermittelte Aufgabenangemessenheit signifikant größer als bei
Gruppe B (M=3.83), t(18)=2.36, p<.05.

% We also generate the table from the data entered
\begin{table}[ht]
  % We want some more white space in this table
  % First the column separation
  \renewcommand{\tabcolsep}{5mm}
  % Then the columns
  \renewcommand{\arraystretch}{1.5}
  \begin{center}

    \pgfplotstabletypeset[zerofill,
      every head row/.style={
        before row={%
          & \multicolumn{2}{c}{\textbf{Gruppe A}} & \multicolumn{2}{c}{\textbf{Gruppe B}}\\
          },
        after row={\hline}},  
      columns/F/.style={
        column type=l,
        column name=\textbf{Faktoren},
        string type},
      columns/M-A/.style={column type=c,column name=\textbf{M}},
      columns/SD-A/.style={column type=c,column name=\textbf{SD}},
      columns/M-B/.style={column type=c,column name=\textbf{M}},
      columns/SD-B/.style={column type=c,column name=\textbf{SD}},
      string replace={AA}{Aufgabenangemessenheit},
      string replace={SB}{Selbstbeschreibungsfähigkeit},
      string replace={EK}{Erwartungskonformität},
      string replace={LF}{Lernförderlichkeit},
      string replace={SK}{Steuerbarkeit},
      string replace={FT}{Fehlertoleranz},
      string replace={IK}{Individualisierbarkeit},]
      {\empiricaldata}

    \caption[Mittelwerte und Standardabweichungen für Gruppen A und B (N=20).]%
      {Mittelwerte (M) und Standardabweichungen (SD) der Faktoren ermittelt mit Hilfe eines Fragebogens ISONORM 9241/110-S \citep{prumper-2012-isonorm-fragebogen} für Gruppen A und B (N=20).}
    \label{table:eval}
  \end{center}
\end{table}%

% % This is how you would describe the same table directly
% \begin{table}[ht]
%   % We want some more white space in this table
%   % First the column separation
%   \renewcommand{\tabcolsep}{5mm}
%   % Then the columns
%   \renewcommand{\arraystretch}{1.5}
%   \begin{center}
%     \begin{tabular}{lrrrr}
%       & \multicolumn{2}{c}{\textbf{Gruppe A}} & \multicolumn{2}{c}{\textbf{Gruppe B}}\\
%       \textbf{Faktoren} & \textbf{M} & \textbf{SD} & \textbf{M} & \textbf{SD} \\
%       \hline
%       Aufgabenangemessenheit       & 5,33 & 0,69 & 5,00 & 0,48 \\
%       Selbstbeschreibungsfähigkeit & 3,92 & 0,96 & 6,33 & 0,35 \\
%       Erwartungskonformität        & 5,88 & 1,10 & 6,10 & 0,56 \\
%       Lernförderlichkeit           & 4,54 & 1,76 & 4,24 & 1,13 \\
%       Steuerbarkeit                & 5,75 & 0,92 & 6,90 & 0,67 \\
%       Fehlertoleranz               & 4,56 & 0,93 & 3,00 & 0,98 \\
%       Individualisierbarkeit       & 4,61 & 0,72 & 5,86 & 0,82 \\
%     \end{tabular}
%     \caption{Mittelwerte (M) und Standardabweichungen (SD) der Faktoren ermittelt mit Hilfe eines Fragebogens ISONORM 9241/110-S \citep{prumper-2012-isonorm-fragebogen} für Gruppen A und B (N=20).}
%     \label{table:eval}
%   \end{center}
% \end{table}%

%%%%%%%%%%%%%%%%%%%%%%%%%%%%%%%%%%%%%%%%%%%%%%%%%%%%%%%%%%%%
\subsection{Diskussion}\label{sec:eva_discuss}
%%%%%%%%%%%%%%%%%%%%%%%%%%%%%%%%%%%%%%%%%%%%%%%%%%%%%%%%%%%%

\imaicomment{
  Eigene Einschätzung der Evaluierung
  
  An dieser Stelle sollten auch kritische Einwände gegen die Methodik erhoben werden
  }

%%%%%%%%%%%%%%%%%%%%%%%%%%%%%%%%%%%%%%%%%%%%%%%%%%%%%%%%%%%%
%%%%%%%%%%%%%%%%%%%%%%%%%%%%%%%%%%%%%%%%%%%%%%%%%%%%%%%%%%%%
\chapter{Schluß}\label{chapter:conclusions}
%%%%%%%%%%%%%%%%%%%%%%%%%%%%%%%%%%%%%%%%%%%%%%%%%%%%%%%%%%%%

\imaicomment{kurze Einleitung zu den Unterkapiteln}

%%%%%%%%%%%%%%%%%%%%%%%%%%%%%%%%%%%%%%%%%%%%%%%%%%%%%%%%%%%%
\section{Zusammenfassung}\label{sec:conc_summary}
%%%%%%%%%%%%%%%%%%%%%%%%%%%%%%%%%%%%%%%%%%%%%%%%%%%%%%%%%%%%

\imaicomment{Darstellung dessen, was erreicht wurde (ca. 1 Seite)}

In dieser Arbeit wurde der QA-Wizard entwickelt, ein System zur Unterstützung von Studierenden bei
der Auswahl geeigneter Dokumentvorlagen für ihre Abschlußarbeiten.

%%%%%%%%%%%%%%%%%%%%%%%%%%%%%%%%%%%%%%%%%%%%%%%%%%%%%%%%%%%%
\section{Einschätzung}\label{sec:conc_assessment}
%%%%%%%%%%%%%%%%%%%%%%%%%%%%%%%%%%%%%%%%%%%%%%%%%%%%%%%%%%%%

\imaicomment{Abschließende Einschätzung der dargestellten Theorien, Methoden und Vorgehensweise sowie eventuell vorgestellter Beispiele}

Auch wenn ein automatisiertes Konfigurieren von Abschlußarbeiten eine Arbeitserleichterung für Studierende darstellen kann so sind bei den vorgestellten Methoden doch eindeutig Grenzen bei der automatischen Erkennung verschiedener Arten von Texten erkennbar.

%%%%%%%%%%%%%%%%%%%%%%%%%%%%%%%%%%%%%%%%%%%%%%%%%%%%%%%%%%%%
% list of figures, tables
%%%%%%%%%%%%%%%%%%%%%%%%%%%%%%%%%%%%%%%%%%%%%%%%%%%%%%%%%%%%

% here we have all the stuff where chapters have no numbers etc.
% you will find a lot of \cleardoublepage and \phantomsection
% commands, these help hyperref.sty to find the right targets
% for hyperlinks
\backmatter

% The list of figures
  \cleardoublepage
  \phantomsection
%   \addcontentsline{toc}{chapter}{Abbildungen}
  \listoffigures

% The list of tables
  \cleardoublepage
  \phantomsection
%   \addcontentsline{toc}{chapter}{Tabellen}
  \listoftables

% list of listings
  \cleardoublepage
  \phantomsection
%   \addcontentsline{toc}{chapter}{Quelltexte}
  \lstlistoflistings  % generated by listings.sty

%%%%%%%%%%%%%%%%%%%%%%%%%%%%%%%%%%%%%%%%%%%%%%%%%%%%%%%%%%%%
%%%%%%%%%%%%%%%%%%%%%%%%%%%%%%%%%%%%%%%%%%%%%%%%%%%%%%%%%%%%
\chapter{Literatur}
% \addcontentsline{toc}{chapter}{Quellen}
%%%%%%%%%%%%%%%%%%%%%%%%%%%%%%%%%%%%%%%%%%%%%%%%%%%%%%%%%%%%

% While it is possible to cite something without putting
% in a reference, the question is, why would you do that?
\nocite{cooper_ea-2014-about_face_4}

\phantomsection
\renewcommand{\bibname}{Literatur}
\bibliographystyle{imai}
\bibliography{bibliography}
% \addcontentsline{toc}{section}{Literatur}

% \phantomsection
% \bibliographystyleweb{imai}
% \bibliographyweb{bibliography}
% \addcontentsline{toc}{section}{Weblinks}

%%%%%%%%%%%%%%%%%%%%%%%%%%%%%%%%%%%%%%%%%%%%%%%%%%%%%%%%%%%%
% documentation of use of generative AI
%%%%%%%%%%%%%%%%%%%%%%%%%%%%%%%%%%%%%%%%%%%%%%%%%%%%%%%%%%%%
\chapter{KI-Nutzung}\label{sec:ai_use}
%%%%%%%%%%%%%%%%%%%%%%%%%%%%%%%%%%%%%%%%%%%%%%%%%%%%%%%%%%%%


  \begin{tabular}{
    >{\small\raggedright\arraybackslash}p{0.25\textwidth}
    >{\small\raggedright\arraybackslash}p{0.25\textwidth}
    >{\small\raggedright\arraybackslash}p{0.4\textwidth}
  }
  \textbf{KI-System}  & \textbf{Verwendung} & \textbf{Beschreibung} \\
  \textbf{\& Version} & \textbf{(Zweck)}    & \textbf{der Nutzung}  \\
  \hline
    Claude 3.7 Sonnet  & 
    Strukturierung der Arbeit & 
    Prompt: ``Schlage eine Gliederung für eine Ausarbeitung zum Thema Audio-Codecs vor'' \\
    GPT-4o (März 2025) &
    Code-Optimierung & 
    Überprüfung meines Python-Codes zur Audioanalyse auf Fehler und Optimierungsmöglichkeiten \\
    Copilot (Version 2025) &
    Code-Generierung &
    Assistenz beim Erstellen von SVG-Grafiken für die Darstellung von Frequenzspektren \\
\end{tabular}


%%%%%%%%%%%%%%%%%%%%%%%%%%%%%%%%%%%%%%%%%%%%%%%%%%%%%%%%%%%%
% we now go into the appendix. basically the appendix is one chapter
% with some sections below. The appendix chapter has no mark, the
% sections alphabetical ones, subsections will have numerical ones
% TODO format lower level sections in appendix
\renewcommand{\thechapter}{\Alph{chapter}}
\renewcommand{\thesection}{\Alph{section}}
\renewcommand{\thesubsection}{\Alph{section}.\arabic{subsection}}
% \setcounter{section}{0} % need to be explicit, since they are not reset in back matter
% \addtocounter{chapter}{1} % help hyperref find the correct section

%%%%%%%%%%%%%%%%%%%%%%%%%%%%%%%%%%%%%%%%%%%%%%%%%%%%%%%%%%%%
%%%%%%%%%%%%%%%%%%%%%%%%%%%%%%%%%%%%%%%%%%%%%%%%%%%%%%%%%%%%
\chapter*{Erklärung über das selbständige Verfassen}\addcontentsline{toc}{chapter}{Erklärung}
\thispagestyle{empty}
%%%%%%%%%%%%%%%%%%%%%%%%%%%%%%%%%%%%%%%%%%%%%%%%%%%%%%%%%%%%
% you did it:) now the only thing left is to sign and deliver it

  {
    Ich versichere hiermit, daß ich die vorstehende Seminararbeit
    selbständig verfaßt und keine anderen als die angegebenen Hilfsmittel
    benutzt habe. Die Stellen der Arbeit, die anderen Werken dem Wortlaut
    oder dem Sinn nach entnommen wurden, habe ich in jedem einzelnen Fall
    durch die Angabe der Quelle bzw. der Herkunft, auch der benutzten
    Sekundärliteratur, als Entlehnung kenntlich gemacht. Dies gilt auch für
    Zeichnungen, Skizzen, bildliche Darstellungen sowie für Quellen aus dem 
    Internet und anderen elektronischen Text- und Datensammlungen und
    dergleichen. Die eingereichte Arbeit ist nicht anderweitig als
    Prüfungsleistung verwendet worden oder in deutscher oder in einer anderen 
    Sprache als Veröffentlichung erschienen. Mir ist bewußt, daß
    wahrheitswidrige Angaben als Täuschung behandelt werden.
  }


\vspace*{4cm}

% FIXME your real name, sign here - do not add the student number
Vorname Zuname

\vspace*{1cm}

Hildesheim, \today

\end{document}
