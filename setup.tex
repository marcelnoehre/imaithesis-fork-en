%%%%%%%%%%%%%%%%%%%%%%%%%%%%%%%%%%%%%%%%%%%%%%%%%%%%%%%%%%%%
% Copyright 2013-2025 by Jörg Cassens <cassens@cs.uni-hildesheim.de>
%
%%%%%%%%%%%%%%%%%%%%%%%%%%%%%%%%%%%%%%%%%%%%%%%%%%%%%%%%%%%%
% LICENSE INFORMATION BEGINS
% 
% This file is part of a collection of sample LaTeX documents
% intended as templates for deliverables in media  informatics
% at the  Institute for Computer Science, University of
% Hildesheim, Germany. 
% 
% As a user, you are granted the following permissions:
% 
% a) If you choose to use one of the templates provided, you are
% permitted to freely copy and modify this file according to your
% needs.
% 
% b) If you wish to modify this file and distribute it with one
% or more templates, you are granted permission to freely copy 
% and modify it according to your needs.
% 
% c) You may also distribute the unchanged file.
% 
% d) This file may also be distributed and/or modified
%
%    1. under the LaTeX Project Public License and/or
%    2. under the GNU Public License.
% 
% Please be aware that this collection of files contains logos
% of the University of Hildesheim. It is your responsibility
% to ensure that you have the necessary rights to use these logos. 
% If you intend to distribute the files outside the University of
% Hildesheim, you must remove the logos.
% 
% LICENSE INFORMATION ENDS
%%%%%%%%%%%%%%%%%%%%%%%%%%%%%%%%%%%%%%%%%%%%%%%%%%%%%%%%%%%%
%
% You should be able to use this file as it is, parts where
% you have to change things or decide on options are marked
% with comments like this:
% FIXME Example: change something here
%
% This file should only need packages available in a
% standard TeXlive install. Input encoding is UTF 8, please
% make sure your editor gets that right. We use pdfLaTeX
% and produce PDF directly, we do not go via DVI.
%
% Happy TeXing:)
%
% Please direct comments, feedback, patches to:
% 
%            cassens@cs.uni-hildesheim.de (jc)
% 
% gitlab-project at:
% 
%   https://www.uni-hildesheim.de/gitlab/mi/imaithesis
% 
%
% ChangeLog
% 2010-09-23  project started for imis at uni of lübeck (jc)
% 2010-09-29  initial release (jc)
% 2010-09-30  bugfix imis.bst (jc)
% 2010-10-08  hacks for listings added (jc)
% 2010-11-16  changed titlepage as in new word template (jc)
% 2011-02-02  experimental workaround listings in german (jc)
% 2011-06-17  changed way of referencing weblinks (jc)
% 2011-07-29  removed broken package ltablex and use plain
%             tabularx. Beware: tabularx does not support
%             tables over several pages (thanks Niels for bug
%             report & Bjoern for debugging) (jc)
% 2011-12-22  changed Betreuer to Wissenschaftliche Begleitung
%             added some color to the listings (jc)
% 2012-03-08  major rework of content (mostly) and style to
%             reflect newest, reworked version of Word template
%             with a lot of new text to help you get going (jc)
% 2012-03-09  reworked listings and captions massively
%             now we rely on ccaptions.sty for a new 
%             imislistings float environment (jc)
% 2012-03-15  graphs and tables from data (jc)
% 2012-04-02  new title page, small changes to glossaries (jc)
% 2012-04-18  fixed small bug in Makefile resulting in a
%             wrong zipfile for publication - devel only (jc)
% 2012-11-09  hex color definitions to uppercase because some
%             installs seem to bark else - thanks Marcin (jc)
% 2012-12-07  Incorporated changes to support project reports
%             as well as theses, use a boolean to switch (jc)
% 2013-06-26  Initial import of project for use in Hildesheim (jc)
% 2013-07-31  Many simplifications, since we do not have to mimic
%             a given style we rely much more on the features of
%             TeX and use less packages
%             Content and form is adapted to what I think would
%             be a good starting point for theses in Hildesheim (jc)
% 2015-10-21  Small adjustments to formatting, preparing to 
%             use uni-gitlab for future development, addded scrhack
%             to make koma work better with listings.sty (jc)
% 2015-12-05  Refactoring to support a seminar template (jc)
% 2016-04-28  Clarified Copyright status and conditions of use (jc)
% 2016-05-05  make imaithesis work with XeLaTeX (jc)
% 2020-11-07  move from IMAI to IfI (jc)
% 2021-02-02  replaced required style scrpage2 which is now
%             obsolete with scrlayer-scrpage, stop-gap measure (jc)
% 2023-05-16  added stuff to mark generative AI (jc)
% 2023-05-19  minor formatting simplifications (jc)
% 2023-05-25  license clarified (jc)
% 2025-04-17  amended information on generative AI (jc)
% 
%%%%%%%%%%%%%%%%%%%%%%%%%%%%%%%%%%%%%%%%%%%%%%%%%%%%%%%%%%%%

% this makes ``References'' a section instead of a chapter
% this has to be done before other packages mess around
% with the definition
\makeatletter
\renewcommand*\bib@heading{%
  \section*{\bibname}%
  \@mkboth{\bibname}{\bibname}%
}
\makeatother

%%%%%%%%%%%%%%%%%%%%%%%%%%%%%%%%%%%%%%%%%%%%%%%%%%%%%%%%%%%%
% Now we can start to load the packages needed
%%%%%%%%%%%%%%%%%%%%%%%%%%%%%%%%%%%%%%%%%%%%%%%%%%%%%%%%%%%%

%%%%%%%%%% General      %%%%%%%%%%
% koma comes with more advanced handling of tocs, scrhack
% makes other packages like listings work with that
% TODO this might cause problems with updated packages
\usepackage{scrhack}
% we want to support both pdfLaTeX and XeLaTeX
% TODO consider luaTeX as well
\usepackage{iftex}
% language settings, output-encoding
% localisation (languages set in document directive):
\usepackage{babel}
% % TODO context sensitive quotes, investigate for biblatex
% \usepackage{csquotes}
% output encoding:
% \usepackage[T1]{fontenc}

%%%%%%%%%% Fonts        %%%%%%%%%%
% TODO validate font choice
% We mainly use clones or derivatives of the Palatino-font
% by Herman Zapf and the Source Pro family of fonts designed
% by Paul D. Hunt.
% All these fonts can be classified as ``dynamic humanist fonts''
% according to Kupferschmid and Willberg.
% 
% If we are using XeLaTeX or luaLaTeX:
% We use a Palatino-clone called TeX Gyre Pagella for most of
% the text, Source Sans Pr for sans-serif and Source Code Pro
% and monospace. The OpenType/TrueType versions of the fonts 
% need to be installed on your computer to use this template.
% On linux-systems, they are usually available through the 
% package  manager (zypper, yum, apt-get). If this is not the
% case or if you use a different operating system such as MacOS
%  or Windows, you might need to  download the fonts manually:
% FIXME install fonts for XeLaTeX and luaLaTeX
% http://www.gust.org.pl/projects/e-foundry/tex-gyre/pagella
% http://www.gust.org.pl/projects/e-foundry/tg-math
% https://github.com/adobe-fonts/source-sans
% https://github.com/adobe-fonts/source-code-pro
% 
% If we are using pdfLaTeX:
% We use a Palatino-clone called TeX Gyre Pagella for most of
% the text, Source Sans Pr for sans-serif and Source Code Pro
% and monospace. All fonts and their support packages should
% be available on any standard TeX distribution.
% 
% we use alwasy UTF-8 input
% Font loading is quite different for pdfLaTeX and XeLaTeX
%   \usepackage{amsmath}
  \ifXeTeX
    % XeLaTeX (system fonts via fontspec)
    \usepackage{fontspec}
    \usepackage{unicode-math}
    \setmainfont {TeX Gyre Pagella}[Scale = 1.05]
    \setromanfont{TeX Gyre Pagella}[Scale = 1.05]
    \setmathfont {TeX Gyre Pagella Math}
    \setsansfont {Source Sans Pro}
    \setmonofont {Source Code Pro}
    \defaultfontfeatures{Ligatures=TeX}
  \else\ifLuaTeX
    % LuaLaTeX (system fonts via fontspec, same syntax as XeTeX)
    \usepackage{fontspec}
    \usepackage{unicode-math}
    \setmainfont {TeX Gyre Pagella}[Scale = 1.05]
    \setromanfont{TeX Gyre Pagella}[Scale = 1.05]
    \setmathfont {TeX Gyre Pagella Math}
    \setsansfont {Source Sans Pro}
    \setmonofont {Source Code Pro}
    \defaultfontfeatures{Ligatures=TeX}
  \else
    % pdfLaTeX (fallback to traditional fonts)
    \usepackage[utf8]{inputenc}
    \usepackage[T1]{fontenc}
    \usepackage[scaled=1.05]{newpxtext}
    \usepackage{newpxmath}
    \usepackage{sourcecodepro}
    \usepackage{sourcesanspro}
  \fi\fi

%%%%%%%%%% Look & Feel  %%%%%%%%%%
% nicer headings/footer:
\usepackage{scrlayer-scrpage}
% typeset URLs nicely:
\usepackage{url}
% extended table environment:
\usepackage{tabularx}
% % colours in tables:
% \usepackage{colortbl}
% % booktabs for nice looking tables
% \usepackage{booktabs}
% used for adapting list environment, e.g. for glossaries:
\usepackage{enumitem}
\usepackage{pifont}
% include source code listings in the text:
\usepackage{listings}

%%%%%%%%%% Graphics     %%%%%%%%%%
% we need to be able to include graphics:
\usepackage{graphicx}
% we use colours in tables etc.:
\usepackage[dvipsnames,usenames]{xcolor}
% TikZ is cool for diagrams:
\usepackage{tikz}
% these 2 packages can be used to generate
% graphs & tables from tabular data:
\usepackage{pgfplots}
\usepackage{pgfplotstable}
% pinning to specific pgfplots version
% \pgfplotsset{compat=1.8}
\pgfplotsset{compat=1.15}
% make text flow around images
% \usepackage{picins}

%%%%%%%%%% Bib          %%%%%%%%%%
% allows us to use author-year styles:
\usepackage{natbib}
% % allows to have a separate bibliography for weblinks (or more):
% \usepackage{multibib} % will be removed in the future

%%%%%%%%%%%%%%%%%%%%%%%%%%%%%%%%%%%%%%%%%%%%%%%%%%%%%%%%%%%%
% colour definitions
%%%%%%%%%%%%%%%%%%%%%%%%%%%%%%%%%%%%%%%%%%%%%%%%%%%%%%%%%%%%
% This colour is used to typeset some info text:
\definecolor{lightgray}{gray}{.5}

% These colours from the Tango project fit well together:
\definecolor{tango_orange_dark}      {HTML}{CE5C00}
\definecolor{tango_orange_medium}    {HTML}{F57900}
\definecolor{tango_orange_light}     {HTML}{FCAF3E}
\definecolor{tango_chocolate_dark}   {HTML}{8F2902}
\definecolor{tango_chocolate_medium} {HTML}{C17D11}
\definecolor{tango_chocolate_light}  {HTML}{E9B96E}
\definecolor{tango_chameleon_dark}   {HTML}{4E9A06}
\definecolor{tango_chameleon_medium} {HTML}{73D216}
\definecolor{tango_chameleon_light}  {HTML}{8AE234}
\definecolor{tango_skyblue_dark}     {HTML}{204A87}
\definecolor{tango_skyblue_medium}   {HTML}{3465A4}
\definecolor{tango_skyblue_light}    {HTML}{729FCF}
\definecolor{tango_plum_dark}        {HTML}{5C3566}
\definecolor{tango_plum_medium}      {HTML}{75507B}
\definecolor{tango_plum_light}       {HTML}{AD7FA8}
\definecolor{tango_scarletred_dark}  {HTML}{A40000}
\definecolor{tango_scarletred_medium}{HTML}{CC0000}
\definecolor{tango_scarletred_light} {HTML}{EF2929}
\definecolor{tango_aluminimum_1}     {HTML}{EEEEEC}
\definecolor{tango_aluminimum_2}     {HTML}{D3D7CF}
\definecolor{tango_aluminimum_3}     {HTML}{BABDB6}
\definecolor{tango_aluminimum_4}     {HTML}{888A85}
\definecolor{tango_aluminimum_5}     {HTML}{555753}
\definecolor{tango_aluminimum_6}     {HTML}{2E3436}
% See http://tango.freedesktop.org/

% These are the colours for the University of Hildesheim:
\definecolor{unihi_red}   {HTML}{D7001C} % Uni 
\definecolor{unihi_ruby}  {HTML}{B4152B} % FB I
\definecolor{unihi_orange}{HTML}{EC870E} % FB II
\definecolor{unihi_blue}  {HTML}{0186BC} % FB III
\definecolor{unihi_green} {HTML}{467A40} % FB IV

%%%%%%%%%% Other        %%%%%%%%%%
% % is already loaded automatically:
% \usepackage{calc}
% % do we want an index?
% \usepackage{makeidx}


% include hyperlinks in PDF and much more:
% FIXME Chose between coloured or black
\usepackage[
  colorlinks=true,
% %   Chose these ``colours'' for grayscale print
%   linkcolor=black,
%   menucolor=black,
%   citecolor=black,
%   urlcolor=black,
% %   Previous deault colours, chose if preferred
%   linkcolor=red!40!black,
%   menucolor=red!40!black,
%   citecolor=blue!50!black,
%   urlcolor=green!40!black,
% % Current default colours
  linkcolor=tango_chocolate_dark,
  menucolor=tango_chocolate_dark,
  citecolor=tango_skyblue_dark,
  urlcolor=tango_plum_dark,
]{hyperref}

% Now all packages are loaded and we can start to modify
% the look and feel.

% TODO recalc typearea
% % letting typearea recalc the layout with all packages loaded
\recalctypearea


%%%%%%%%%%%%%%%%%%%%%%%%%%%%%%%%%%%%%%%%%%%%%%%%%%%%%%%%%%%%
% Set up header and footer
%%%%%%%%%%%%%%%%%%%%%%%%%%%%%%%%%%%%%%%%%%%%%%%%%%%%%%%%%%%%

\pagestyle{scrheadings}
\clearpairofpagestyles
\renewcommand{\headfont}{\normalfont\rmfamily\itshape}
\ohead{\headmark}
\ofoot[\pagemark]{\pagemark}

%%%%%%%%%%%%%%%%%%%%%%%%%%%%%%%%%%%%%%%%%%%%%%%%%%%%%%%%%%%%
%   lists of... and captions
%%%%%%%%%%%%%%%%%%%%%%%%%%%%%%%%%%%%%%%%%%%%%%%%%%%%%%%%%%%%

% for listings, we define the names used in the captions
% and in the list of listings
% we also take the -verzeichnis out of the list of
% figures and list of tables
% TODO differences in defined languages/babel; Text formating
\AtBeginDocument{%
  \makeatletter
    \addto\captionsngerman{%
      \def\lstlistingname{Quelltext}%
      \def\lstlistlistingname{Quelltexte}%
%       \renewcommand{\listfigurename}{\normalfont\rmfamily Abbildungen}%
%       \renewcommand{\listtablename}{\normalfont\rmfamily Tabellen}%
      \renewcommand{\listfigurename}{Abbildungen}%
      \renewcommand{\listtablename}{Tabellen}%
    }%
  \makeatother
}
% % for other than KOMA classes, the following may have
% % to be used instead:
% \renewcommand*{\lstlistlistingname}{Quelltexte}
% \renewcommand*{\lstlistingname}{Quelltexte}

% captions should be set in a small font size
\addtokomafont{caption}{\footnotesize}
% \addtokomafont{captionlabel}{\bfseries}

%%%%%%%%%%%%%%%%%%%%%%%%%%%%%%%%%%%%%%%%%%%%%%%%%%%%%%%%%%%%
%   General changes to formatting
%%%%%%%%%%%%%%%%%%%%%%%%%%%%%%%%%%%%%%%%%%%%%%%%%%%%%%%%%%%%

% First some basic styling for listings
% FIXME you will probably not use SPARQL
\lstset{
  keywordstyle=\color{red!40!black},
  commentstyle=\itshape\color{green!40!black},
  stringstyle=\color{blue!50!black},
  basicstyle=\ttfamily,
  frame=single,
%   language=xml,
  language=sparql,
  captionpos=b,
  nolol=false
  }

% this is the KOMA-way of changing titles.
\setkomafont{sectioning}{\rmfamily\bfseries}
  
% we want an url to be printed in the standard font
\urlstyle{same}

%%%%%%%%%%%%%%%%%%%%%%%%%%%%%%%%%%%%%%%%%%%%%%%%%%%%%%%%%%%%
% Set up citation styles
%%%%%%%%%%%%%%%%%%%%%%%%%%%%%%%%%%%%%%%%%%%%%%%%%%%%%%%%%%%%

% Deprecated: we will not make use of multibib styles
% % we use multibib.sty to generate a weblinks bibliography,
% % define more as you see fit (Software, RFCs, Images)
% \newcites{web}{Weblinks}
% % FIXME If you use several bibliographies, you need to run
% % all of them through bibTeX. This can be achieved with the
% % following script (for Unix, Mac, and Linux):
% #!/bin/bash
% for file in *.aux ; do
%   bibtex `basename $file .aux`
% done

%%%%%%%%%%%%%%%%%%%%%%%%%%%%%%%%%%%%%%%%%%%%%%%%%%%%%%%%%%%%
% float control
%%%%%%%%%%%%%%%%%%%%%%%%%%%%%%%%%%%%%%%%%%%%%%%%%%%%%%%%%%%%
% more sensible defaults for how much space floats can take
% up on pages with text:
% http://www.tug.org/texmf-dist/doc/generic/FAQ-en/html/FAQ-floats.html
\setcounter{topnumber}{2}
\setcounter{bottomnumber}{9}
\setcounter{totalnumber}{20}
\setcounter{dbltopnumber}{9}
\renewcommand{\topfraction}{0.85}
\renewcommand{\bottomfraction}{0.7}
\renewcommand{\textfraction}{0.15}
\renewcommand{\floatpagefraction}{0.7}
\renewcommand{\dbltopfraction}{.7}
\renewcommand{\dblfloatpagefraction}{.7}

%%%%%%%%%%%%%%%%%%%%%%%%%%%%%%%%%%%%%%%%%%%%%%%%%%%%%%%%%%%%
% some commands necessary for the template
%%%%%%%%%%%%%%%%%%%%%%%%%%%%%%%%%%%%%%%%%%%%%%%%%%%%%%%%%%%%
% the \imaicomment command outputs some hints about what to
% write about in the different parts of the thesis,
% you should either delete them all or redefine this
% command to an empty one for the final version.
%
% This command can be quite useful to capture comments and
% questions you want to deal with later.
% \newcommand{\imaicomment}[1]{{\itshape\color{lightgray} #1}}
\newcommand{\imaicomment}[1]{{\itshape\color{tango_aluminimum_4} #1}}

% The next two commands allow you to easily 
\newcommand{\migenaiverb}[1]{{\color{tango_orange_dark} #1}}
\newcommand{\migenaimod}[1]{{\color{tango_chameleon_dark} #1}}
\newcommand{\migenaiused}{{\color{tango_chameleon_dark}\raisebox{.125ex}{\scalebox{0.8}{\ding{70}}}\hspace{-0.2em}\raisebox{.8ex}{\scalebox{0.5}{\ding{70}}}\hspace{-0.4em}\raisebox{-0.1ex}{\scalebox{0.5}{\ding{70}}}}}


% experimental support for longer text in teletype-style
% (\texttt{}) with added linebreaks and hyphenation
% adapted from http://tex.stackexchange.com/questions/299/how-to-get-long-texttt-sections-to-break
\newcommand*{\imaitexttt}[1]{\texttt{%
    \fontdimen2\font=0.4em% interword space
    \fontdimen3\font=0.2em% interword stretch
    \fontdimen4\font=0.1em% interword shrink
    \fontdimen7\font=0.1em% extra space
    \hyphenchar\font=`\-% allowing hyphenation
    #1}}

%%%%%%%%%%%%%%%%%%%%%%%%%%%%%%%%%%%%%%%%%%%%%%%%%%%%%%%%%%%%
% some options for debugging
%%%%%%%%%%%%%%%%%%%%%%%%%%%%%%%%%%%%%%%%%%%%%%%%%%%%%%%%%%%%

% \usepackage{showframe}
% \usepackage[sfdefault]{classico}
% \usepackage{lineno}
% \usepackage{lipsum}

%%%%%%%%%%%%%%%%%%%%%%%%%%%%%%%%%%%%%%%%%%%%%%%%%%%%%%%%%%%%
% now that setup is complete, we continue in the main files
%%%%%%%%%%%%%%%%%%%%%%%%%%%%%%%%%%%%%%%%%%%%%%%%%%%%%%%%%%%%
